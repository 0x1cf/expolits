\documentclass[t]{beamer}
\usepackage{listings}
\usepackage{color}
\usepackage{graphicx}
\usepackage[adobefonts, noindent]{ctex}

\lstset{
	language=C++,
	columns=fullflexible,
	basicstyle=\footnotesize\ttfamily,
	%numbers=left,
	%numbersep=6pt,
	frame=L,
	xleftmargin=0pt,
	showstringspaces=false,
	keywordstyle=\color{blue}\ttfamily,
	stringstyle=\color{red}\ttfamily,
	commentstyle=\color{magenta}\ttfamily,
	morecomment=[l][\color{magenta}]{\#}
	tabsize=2,
   	keepspaces=true,
}


\makeatletter
\def\verbatim@font{\linespread{1}\normalfont\ttfamily}
\makeatother
\begin{document}
\title{CVE-2013-2072}
\subtitle{\textit{从溢出的角度}}
\author{张帆}
\date{2013-12-24}

% titlepage
\frame{\titlepage}

% TOC
\frame{\frametitle{Outline}
\tableofcontents
}

\section{Heap Overflow}
\subsection{vulnerable code}
\begin{frame}[fragile]
\frametitle{\kaishu 溢出现场}

\begin{lstlisting}[basicstyle=\footnotesize]
225     xc_cpumap_t cpumap; 
239     cpumap = xc_cpumap_alloc(self->xc_handle); 
245         for ( i = 0; i < PyList_Size(cpulist); i++ )
246         {
247             long cpu = PyInt_AsLong(PyList_GetItem(cpulist, i));
248             /*if ( cpu < 0 || cpu >= nr_cpus )
249             { return ...
254             }*/
255             cpumap[cpu / 8] |= 1 << (cpu % 8);  /* overflow */
256         }
257     }
\end{lstlisting}

\begin{enumerate}
\item 写入\texttt{cpumap[cpu/8]}时没有检查边界
\item \texttt{cpu}是外部参数,可以通过\texttt{xm vcpu-pin}传入
\end{enumerate}
\end{frame}

\begin{frame}[fragile]{Heap Overflow}

准确来说,是``堆''缓存溢出。

\begin{lstlisting}
354 typedef uint8_t *xc_cpumap_t; // cpumap ~ uint8_t *

 86 xc_cpumap_t xc_cpumap_alloc(xc_interface *xch){
 88     int sz;
 90     sz = xc_get_cpumap_size(xch);
 91     if (sz == 0)
 92         return NULL;
 93     return calloc(1, sz); // allocate mem from heap
 94 }
\end{lstlisting}
cpumap是堆中的buffer,大小为sz。cpumap即为\texttt{uint8\_t *}。
\end{frame}

\subsection{capability}
\begin{frame}[fragile]
\frametitle{\kaishu 能做什么}
\begin{lstlisting}
long cpu = cpulist[i];
//cpumap[cpu / 8] |= 1 << (cpu % 8);
*(cpumap + cpu / 8) |= 1 << (cpu % 8);
\end{lstlisting}


实际上,上面的赋值将\texttt{cpumap}之后的\textbf{第cpu个bit置为1}。

因此,通过操纵cpu,可以任意修改cpumap(堆中)。
\end{frame}

\subsection{DOS}
\begin{frame}[fragile]
\frametitle{\kaishu 实现了DOS}
\begin{verbatim}
> xm vcpu-pin 0 0 999999
\end{verbatim}
{\scriptsize\begin{verbatim}
Error: Unable to connect to xend: Connection refused. Is xend running?
\end{verbatim}}

Log\par

{\tiny
\begin{verbatim}
Dec 25 22:29:36 xen py_xc[8099]: cpumap at 0x7f1530006900, len=1
Dec 25 22:29:36 xen py_xc[8099]: rsp=  0x7f15495d5ae0
Dec 25 22:29:36 xen py_xc[8099]: __heap-2             0x70
Dec 25 22:29:36 xen py_xc[8099]: __heap-1             0x24
Dec 25 22:29:36 xen py_xc[8099]: Error Detection: cpu=999999
Dec 25 22:29:36 xen py_xc[8099]: setting cpumap (0x7f1530025147.7), vcpu=0
----------------------------
[2013-12-25 22:29:36 8098] CRITICAL (SrvDaemon:232) Xend died due to signal 11!  Restarting it.
\end{verbatim}}
\end{frame}

\section{Heap -- glibc}
\begin{frame}[fragile]
\frametitle{Heap implementation}
\begin{description}
\item[什么是堆]
\begin{itemize}
\item 堆是可供程序运行时申请的内存空间
\item 堆的存在不受限于函数的executing scope(不同于stack)
\end{itemize}
\item[堆的实现]
\begin{itemize}
  \item 堆在地址低段,向高地址增长
  \item Linux中的堆实现(\texttt{calloc(), free()})是\texttt{glibc}的一部分
\end{itemize}
\end{description}
\textbf{堆溢出}是非常常见的安全漏洞,为了利用heap overflow,需要了解堆。
\end{frame}

\subsection{struct -- chunk}
\begin{frame}[fragile]{\kaishu 堆的结构}{\fangsong chunk数据结构}

堆是一整块连续的内存,由若干chunk组成。Allocated chunk的数据结构如下。
{\scriptsize
\begin{verbatim}
    chunk-> +-+-+-+-+-+-+-+-+-+-+-+-+-+-+-+-+-+-+-+-+-+-+-+-+-+-+-+-+-+-+-+-+
            |             Size of previous chunk, if allocated            | |
            +-+-+-+-+-+-+-+-+-+-+-+-+-+-+-+-+-+-+-+-+-+-+-+-+-+-+-+-+-+-+-+-+
            |             Size of chunk, in bytes                       |M|P|
      mem-> +-+-+-+-+-+-+-+-+-+-+-+-+-+-+-+-+-+-+-+-+-+-+-+-+-+-+-+-+-+-+-+-+
            |             User data starts here...                          |
            |                                                               |
nextchunk-> +-+-+-+-+-+-+-+-+-+-+-+-+-+-+-+-+-+-+-+-+-+-+-+-+-+-+-+-+-+-+-+-+
            |             Size of chunk                                     |
            +-+-+-+-+-+-+-+-+-+-+-+-+-+-+-+-+-+-+-+-+-+-+-+-+-+-+-+-+-+-+-+-+
\end{verbatim}}
\end{frame}

\begin{frame}[fragile]
\frametitle{\kaishu free chuck组成双向链表}
\begin{lstlisting}[basicstyle=\scriptsize]
struct malloc_chunk {
  INTERNAL_SIZE_T      prev_size;  /* Size of previous chunk (if free).  */
  INTERNAL_SIZE_T      size;       /* Size in bytes, including overhead. */

  struct malloc_chunk* fd;         /* double links -- used only if free. */
  struct malloc_chunk* bk;

  /* Only used for large blocks: pointer to next larger size.  */
  struct malloc_chunk* fd_nextsize; /* double links -- used only if free. */
  struct malloc_chunk* bk_nextsize;
};
\end{lstlisting}

双向链表的操作可以被利用。

\end{frame}

\subsection{ops -- traversal}
\begin{frame}[fragile]
\frametitle{\kaishu 堆的遍历}
堆溢出可利用的根源在\textbf{基于``size''的遍历操作},比如:
\begin{lstlisting}
// get chunk at offset
1323 #define chunk_at_offset(p, s) \
    ((mchunkptr)(((char*)(p)) + (s)))
// get next chunk by size
1437 #define next_chunk(p) \
    ((mchunkptr)( ((char*)(p)) + ((p)->size & ~SIZE_BITS) ))
\end{lstlisting}
\end{frame}

\subsection{overflow -- overwrite metadata}
\begin{frame}
\frametitle{\kaishu 为什么溢出可以被利用}
覆盖metadata(特别是prev-size, size)可以控制堆遍历的次序。
\begin{figure}
\centering
\begin{tabular}{|c|}
  \hline
  prev-size \\ \hline
  size \\ \hline
  user data ( overflow !) \\ \hline
  ... \\ \hline
  prev-size \\ \hline
  size \\ \hline
  user data \\ \hline
\end{tabular}
\end{figure}
\end{frame}

\section{Exploiting free()}
\begin{frame}[fragile]
\frametitle{free(void *p)}
\texttt{free(void* P)}的过程
\begin{enumerate}
  \item 安全性检查(*)
  \item 标记\texttt{P->malloc\_chunk}中的位
  \item 将相邻的free chunks合并(双向链表的删除)(**)
\end{enumerate}
合并分两种:前向、后向。分别可以利用。
\end{frame}

\subsection{forward consolidate}
\begin{frame}[fragile]
\frametitle{\kaishu 前向合并}
\begin{lstlisting}
_int_free(mstate av, mchunkptr p, int have_lock)
3892     if (!prev_inuse(p)) {      /* (1) */
3893       prevsize = p->prev_size; /* (2) */
3894       size += prevsize;
3895       p = chunk_at_offset(p, -((long) prevsize));  /* (3) */
3896       unlink(p, bck, fwd);     /* (4) */
3897     }
\end{lstlisting}
\begin{enumerate}
  \item 检查前一个chunk是否free
  \item 得到前一个chunk的size
  \item 将指针前移prev\_size
  \item 从双向链表中删除
\end{enumerate}
\end{frame}

\begin{frame}[fragile]
\frametitle{\kaishu 利用双向链表的删除}
我们可以通过溢出将prev\_size覆盖,从而将p指向我们想要的地方。而之后的双向链表删除,可以利用p实现对任意内存的修改。
\begin{lstlisting}
1424 #define unlink(P, BK, FD) {
1425   FD = P->fd;   \    // FD = A
1426   BK = P->bk;   \    // BK = B
1430   FD->bk = BK;  \    // *(A+12) = B;
1431   BK->fd = FD;  \    // *(B+8) = A;
1451 }
\end{lstlisting}

总之,选取两个合法地址A,B,可以实现任意赋值

\texttt{*(A+12) = B;*(B+8) = A;}
\end{frame}

\subsection{backward consolidate}
\begin{frame}[fragile]
\frametitle{\kaishu 利用后向合并}

\begin{lstlisting}
3856    nextchunk = chunk_at_offset(p, size); // 1
3901    nextinuse = inuse_bit_at_offset(nextchunk, nextsize); // 2'1
4084    if (!nextinuse) { // 2'2
4085        unlink(nextchunk, bck, fwd); // 3
4086        size += nextsize;
4087    } else
4088        clear_inuse_bit_at_offset(nextchunk, 0);
\end{lstlisting}
\begin{enumerate}
  \item 得到下一个chunk的指针
  \item 判断下一个chunk是否free
  \item 双向链表删除
\end{enumerate}
\end{frame}

\section{Exploiting}
\begin{frame}[fragile]
\frametitle{\kaishu 情况}
\begin{itemize}
  \item 能利用的只有一个chunk(cpumap),很难再calloc一个
  \item 但是size可以调整(max\_vcpu)
  \item free(cpumap)也不能变
\end{itemize}

因此利用前向合并比较困难,只能利用后向合并。

\end{frame}

\begin{frame}[fragile]{\kaishu 构造chunk,利用后向合并}
\begin{itemize}
  \item 构造2个额外的chunk
  \item 调用free(cpumap),出发后向合并
\end{itemize}
\begin{lstlisting}
//sizeof(size_t) == 8
\end{lstlisting}
{\small \begin{verbatim}
+--------------------------------------------------------+
|prev_sz| size | ... |prev_sz| size | A | B |prev_sz|size|
|(0x70) | 0x24 |     |       |    |?|   |   |       |  |0|
+--------------------------------------------------------+
               ^     ^
               |     |
             cpumap  next_chunk
\end{verbatim}}
\begin{itemize}
  \item 目前正在看\texttt{\_int\_free(..)}源码,研究如何绕过安全检查。
\end{itemize}
\end{frame}

\begin{frame}{Discussion}
?
\end{frame}

\end{document}
